\documentclass[letterpaper]{report}

\usepackage[utf8]{inputenc} % allow utf-8 input
\usepackage[T1]{fontenc}    % use 8-bit T1 fonts
\usepackage{hyperref}       % hyperlinks
\usepackage{url}            % simple URL typesetting
\usepackage{booktabs}       % professional-quality tables
\usepackage{amsfonts}       % blackboard math symbols
\usepackage{nicefrac}       % compact symbols for 1/2, etc.
\usepackage{microtype}      % microtypography
\usepackage{xcolor}         % colors
\usepackage{subcaption,graphicx,listings,algorithm,amsmath}

\title{The Structure of the Emergent Language Revolution}


% The \author macro works with any number of authors. There are two commands
% used to separate the names and addresses of multiple authors: \And and \AND.
%
% Using \And between authors leaves it to LaTeX to determine where to break the
% lines. Using \AND forces a line break at that point. So, if LaTeX puts 3 of 4
% authors names on the first line, and the last on the second line, try using
% \AND instead of \And before the third author name.


\author{%
  Brendon Boldt\\
  Language Technologies Institute\\
  Carnegie Mellon University\\
  Pittsburgh, PA 15213 \\
  \texttt{bboldt@cs.cmu.edu} \\
}

\captionsetup[table]{skip=\baselineskip}

\newcommand\cmr[1]{{\color{red}[#1]}}
\newcommand\cmg[1]{{\color{gray}[#1]}}
\newcommand\drm[1]{{\color{blue}[#1]}}

\newcommand\theProcess{\textsc{FiLex}}
\newcommand\Nodyn{\textsc{NoDyn}}
\newcommand\Sig{\textsc{Sig}}
\newcommand\Nav{\textsc{Nav}}
\newcommand\Recon{\textsc{Recon}}

\begin{document}


\maketitle

\tableofcontents

\setcounter{chapter}{-1}
\chapter{Planning}

\section{Publications}
\begin{itemize}
    \item{} \fbox{submitted} Case study in scientific methods: \textsc{FiLex}
    \item{} \fbox{todo} The goals of emergent language; science--engineering distinction
    \item{} \fbox{nsf} \fbox{todo} data-driven metric
    \item{} \fbox{nsf} \fbox{todo} characteristic-driven metric
    \item{} \fbox{todo} research program for emergent language
    \item{} \fbox{potential} Recommendations for scientific methods
    \item{} \fbox{potential} Recommendations for engineering methods
    \item{} \fbox{nsf} \fbox{potential} Case study in engineering
\end{itemize}

\section{Committee}
\begin{itemize}
    \item David: computational linguistics, language change
    \item Yonatan: embodied NLP, robotics
    \item \fbox{WIP} Philosophy science from Philosophy of Department
    \item \fbox{WIP} External
        \begin{itemize}
        \item Marco Baroni: contemporary emergent language
        \item Kenny Smith: contemporary/classical emergent language, language evolution
        \item Simon Kirby: contemporary/classical emergent language, language evolution
        \end{itemize}

\end{itemize}

\section{NSF grant proposal}

\chapter{The Emergent Language Revolution}
\section{What emergent language is}
\section{What emergent language can do}
\section{The prerequisites of a revolution}

\chapter{The Goals of Emergent Language Research}
\section{Scientific goals}
\section{Engineering goals}
\section{Measuring progress}

\chapter{Emergent Language as Science}
\section{Methods of science}
\section{Case study: Fixed Lexicon Stochastic Process}

\chapter{Emergent Language as Engineering}
\section{Methods of engineering}
\section{Case study: ???}

\chapter{Measuring Progress}
\section{The metric}
\section{Characteristic-driven metrics}
\section{Data-driven metrics}

\chapter{A Research Program for Emergent Language}

\end{document}

%%% Local Variables:
%%% mode: latex
%%% TeX-master: t
%%% End:
