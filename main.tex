\documentclass[letterpaper]{report}

\usepackage[utf8]{inputenc} % allow utf-8 input
\usepackage[T1]{fontenc}    % use 8-bit T1 fonts
\usepackage{hyperref}       % hyperlinks
\usepackage{url}            % simple URL typesetting
\usepackage{booktabs}       % professional-quality tables
\usepackage{amsfonts}       % blackboard math symbols
\usepackage{nicefrac}       % compact symbols for 1/2, etc.
\usepackage{microtype}      % microtypography
\usepackage{xcolor}         % colors
\usepackage{subcaption,graphicx,listings,algorithm,amsmath}

\title{The Structure of the Emergent Language Revolution}


% The \author macro works with any number of authors. There are two commands
% used to separate the names and addresses of multiple authors: \And and \AND.
%
% Using \And between authors leaves it to LaTeX to determine where to break the
% lines. Using \AND forces a line break at that point. So, if LaTeX puts 3 of 4
% authors names on the first line, and the last on the second line, try using
% \AND instead of \And before the third author name.


\author{%
  Brendon Boldt\\
  Language Technologies Institute\\
  Carnegie Mellon University\\
  Pittsburgh, PA 15213 \\
  \texttt{bboldt@cs.cmu.edu} \\
}

% \captionsetup[table]{skip=\baselineskip}

\newcommand\bjb[1]{{\color{blue}[#1]}}
\newcommand\cmg[1]{{\color{gray}[#1]}}
\newcommand\drm[1]{{\color{red}[#1]}}
% \newcommand\badge[1]{\fbox{\footnotesize\texttt{#1}}}


\begin{document}


\maketitle

\tableofcontents

\setcounter{chapter}{-1}
\chapter{Planning}

\section{Publications}
\begin{itemize}
    \item{} \cmg{submitted} Case study in scientific methods: \textsc{FiLex}
    \item{} \cmg{todo} The goals of emergent language; science--engineering distinction
    \item{} \cmg{nsf} \cmg{todo} data-driven metric
    \item{} \cmg{nsf} \cmg{todo} characteristic-driven metric
    \item{} \cmg{todo} research program for emergent language
    \item{} \cmg{potential} Recommendations for scientific methods
    \item{} \cmg{potential} Recommendations for engineering methods
    \item{} \cmg{nsf} \cmg{potential} Case study in engineering
\end{itemize}

\section{Committee}
\begin{itemize}
    \item David: computational linguistics, language change
    \item Yonatan: embodied NLP, robotics
    \item \cmg{WIP} Philosophy science from Philosophy of Department
    \item \cmg{WIP} External
        \begin{itemize}
        \item Marco Baroni: contemporary emergent language
        \item Kenny Smith: contemporary/classical emergent language, language evolution
        \item Simon Kirby: contemporary/classical emergent language, language evolution
        \end{itemize}

\end{itemize}

\section{NSF grant proposal}
\bjb{%
I am currently thinking that the NSF grant proposal should focus on (or at least start with) the publications mentioned above tagged with \cmg{nsf}.
These publications are the most concrete and lean towards machine learning and computational linguistics more than philosophy of science.
The general idea, then would be that the non-NSF papers would be completed primarily before the NSF grant kicks in.%
}

\chapter{The Emergent Language Revolution}
\section{What emergent language is}
Emergent language is the study of simulations of how language arises \emph{de novo}.
This is distinguished from two related topics: the evolution of language and computational models of diachronic language change.
The former is similar in subject matter, the origin of language, but relies on different methods such as the comparative method, animal psychology, or anthropology.
The latter is similar in its methods, simulations and computational models, but typically looks at language change starting with a fully-fledged human language rather than starting from nothing.
The combination these topics and techniques make emergent language its own distinct field with unique goals and challenges.

This field lives also under the names ``language emergence'' and ``emergent communication.''
The former name is simply less common, but the latter implies a slightly different subject matter.
Although, speaking of ``communication'' avoids the philosophical mire of precisely what makes a language, speaking of ``language'' better emphasizes the fact that the distinctly human phenomenon of language is what the field mostly cares about.
For this reason, we will favor ``emergent language'' in this work.

\subsection{Classical approaches to emergent language}
There are some approaches to emergent language, primarily before the late 2010's, that use methods apart from deep learning, such as mathematical models, shallow neural networks, or laboratory experiments \cmg{cite}.
These we term ``classical'' approaches in analogy to \emph{classical} machine learning.
This research serves as historical precedent for deep learning-based emergent language research and is by no means obsolete as an approach, but this work will focus on deep learning-based approaches due to a larger perceived potential.

\subsection{Deep learning approaches to emergent language}
The advent of deep, general-purpose neural networks has set the stage for the primary approach of emergent language which this work will address.
The skeleton of this approach realizing language-learning agents as neural networks which are trained using multi-agent reinforcement learning and stochastic gradient descent.
What makes this approach distinct from classical approaches is that general-purpose neural network dramatically decrease the constraining inductive biases which would otherwise trivialize the truly \emph{emergent} properties of language learned by the agents.

\subsection{Defining ``emergent''}

\section{What emergent language can do}
\subsection{Self-play and emergent behavior}
\cmg{OpenAI hide and seek}
\cmg{OpenAI Five}
\cmg{AlphaZero}
\section{The prerequisites of a revolution}

\chapter{The Goals of Emergent Language Research}
\section{Scientific goals}
\section{Engineering goals}
\section{Measuring progress}

\chapter{Emergent Language as Science}
\section{Methods of science}
\section{Current literature}
\section{Case study: Fixed Lexicon Stochastic Process}

\chapter{Emergent Language as Engineering}
\section{Methods of engineering}
\section{Current literature}
\section{Case study: ???}

\chapter{Measuring Progress}
\section{What a metric is}
\section{Characteristic-driven metrics}
\section{Data-driven metrics}

\chapter{A Research Program for Emergent Language}

\end{document}

%%% Local Variables:
%%% mode: latex
%%% TeX-master: t
%%% End:
