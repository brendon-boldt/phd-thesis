\setcounter{chapter}{-1}
\chapter{Planning}

\section{Publications}
\begin{itemize}
    \item{} \cmg{submitted} Case study in scientific methods: \textsc{FiLex}
    \item{} \cmg{todo} The goals of emergent language; science--engineering distinction
    \item{} \cmg{nsf} \cmg{todo} data-driven metric
    \item{} \cmg{nsf} \cmg{todo} characteristic-driven metric
    \item{} \cmg{todo} research program for emergent language
    \item{} \cmg{potential} Recommendations for scientific methods
    \item{} \cmg{potential} Recommendations for engineering methods
    \item{} \cmg{nsf} \cmg{potential} Case study in engineering
\end{itemize}

\section{Committee}
\begin{itemize}
    \item David: computational linguistics, language change
    \item Yonatan: embodied NLP, robotics
    \item \cmg{WIP} Philosophy science from Philosophy of Department
    \item \cmg{WIP} External
        \begin{itemize}
        \item Marco Baroni: contemporary emergent language
        \item Kenny Smith: contemporary/classical emergent language, language evolution
        \item Simon Kirby: contemporary/classical emergent language, language evolution
        \end{itemize}

\end{itemize}

\section{NSF grant proposal}
\bjb{%
I am currently thinking that the NSF grant proposal should focus on (or at least start with) the publications mentioned above tagged with \cmg{nsf}.
These publications are the most concrete and lean towards machine learning and computational linguistics more than philosophy of science.
The general idea, then would be that the non-NSF papers would be completed primarily before the NSF grant kicks in.%
}

