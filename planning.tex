\setcounter{chapter}{-1}
\chapter{Planning}

\section{Thesis Structure}
I am having second thoughts on this specific structure.
Namely, looking at the individual goals of emergent language research, each has its own challenges that (1) cookie-cutter recommendations are of little help to and (2) probably do not need methodology recommendations when people clearly pursue that goal.
The issue in the literature seems more often that these goals are not even clarified in the first place, and while I can certainly provide a list of clear goals (and I still hope to), it is not as if I can tell people what problems they are actually trying to solve when they go write a paper (unless I'm one of their reviewers).

For example, if people were trying to study the evolution of language with emergent language, I suspect they would already have a solid background in the evolution of language in which case they would be well-attuned to what sort of experiments and data would actually be helpful to that discipline.
With regards to the engineering goal of using emergent language as an alternative data paradigm, I think that goal could derive benchmarks from extant ones in machine translation and other large NLP tasks to which it could be applied.
For goals like robust multi-agent communication or cooperating with humans, I think it would be important to look to emergent language's big brother reinforcement learning; what I think we would find there is that there aren't very many benchmarks or widely-used metrics which measure overall solution quality.

This is not at all to say this thesis' tack is dead in the water.
Rather, I think it should steer away from making general methodological recommendations and focus on the more focused line of reasoning:
\begin{enumerate}
    \item For all of these wonderful applications of emergent language, emergent language needs to be sufficiently advanced/sophisticated/human-like.
    \item This requires a way both to measure this desideratum (cf., metrics, benchmarks) as well as reason about how to design it and improve upon it in practice (cf., theoretical models).
    \item Much of the literature seems to be poking around at this idea of general progress but is doing a poor job of it due to lack metrics, benchmarks, models, etc.\@ (or so I think).
\end{enumerate}



\section{Committee}
\begin{itemize}
    \item David: computational linguistics, language change
    \item Yonatan: embodied NLP, robotics
    \item \cmg{WIP} Philosophy science from Philosophy Department
    \item \cmg{WIP} External
        \begin{itemize}
        \item Marco Baroni: contemporary emergent language
        \item Kenny Smith: contemporary/classical emergent language, language evolution
        \item Simon Kirby: contemporary/classical emergent language, language evolution
        \end{itemize}
\end{itemize}

\section{Publications}
\begin{itemize}
    \item{} \cmg{submitted} Case study in scientific methods: \textsc{FiLex}
    \item{} \cmg{todo} The goals of emergent language; science--engineering distinction
    \item{} \cmg{nsf} \cmg{todo} data-driven metric
    \item{} \cmg{nsf} \cmg{todo} characteristic-driven metric
    \item{} \cmg{todo} research program for emergent language
    \item{} \cmg{potential} Recommendations for scientific methods
    \item{} \cmg{potential} Recommendations for engineering methods
    \item{} \cmg{nsf} \cmg{potential} Case study in engineering
\end{itemize}

\section{NSF grant proposal}
\bjb{%
I am currently thinking that the NSF grant proposal should focus on (or at least start with) the publications mentioned above tagged with \cmg{nsf}.
These publications are the most concrete and lean towards machine learning and computational linguistics more than philosophy of science.
The general idea, then would be that the non-NSF papers would be completed primarily before the NSF grant kicks in.%
}

\drm{%
  You should not be afraid of doing work relevant to a grant before the grant begins. In fact, work in the area of the grant shows that the proposed work is possible and that you are capable of carrying it out. I strongly suggest working upon some epirical/ML work up front.
}

\paragraph{Title}
Improving Low-Resource NLP with Emergent Language

\paragraph{Proposed Work}
\begin{itemize}
    \item Review paper on goals of emergent language research
    \item Data-driven metric of how ``human-like'' an emergent language is
        \begin{itemize}
        \item Semantics-augmented emergent language comparison
        \end{itemize}
    \item Linguistic feature-focused metric of how ``human-like'' an emergent language is
        \begin{itemize}
        \item Does linguistic proximity between a human and emergent language correlate with pretraining quality?
        \item Do surface features or underlying semantics matter more for pretraining?
        \end{itemize}
    \item Benchmark of emergent language
        \begin{itemize}
        \item This might require an analysis on various metrics in emergent language
        \end{itemize}
    \item Research program
\end{itemize}

%% NOTES %%
% - How do the philosophical aspects of this thesis apply to other fields
% - Reverse game theory and mechanism design

\paragraph{Broader Impact}
\drm{I think these are all good, but you need to think about them less tentatively.}
\begin{itemize}
    \item \cmg{direct} Assist researchers in emergent language
    \item \cmg{direct} Learn about inductive biases of neural networks?
    \item \cmg{direct} Measure the understandability of natural communication in multi-agent systems?
    \item \cmg{indirect} Alternative data source for low-resource language \drm{or domain?}
    \item \cmg{indirect} Alternative multimodal data source
\end{itemize}
